\documentclass[12pt]{article}
% 16 03 2017

\usepackage{amsmath}

\newcommand{\E}{\textbf{E}}
\newcommand{\N}{\textbf{Norm}}

\setlength\parskip{1em}
\setlength\parindent{0em}

\begin{document}
{\Large Mar 2017}

\section{Divergence}
So there a couple of different concepts that relate to our weights:

\begin{itemize}
  \item Kullback-Leibler divergence
  \item Mutual information
  \item Likelihood-ratio testing
  \item relative entropy
\end{itemize}

The Kullback-Leibler divergence 
$$ D_\mathrm{KL}(P, Q) = \sum_i P(i) \log(\frac{P(i)}{Q(i)})$$
describes the: information gained by using $P$ instead of $Q$; or the relative entropy of $P$ with respect to $Q$.

For a multivariate normal distribution we often consider the divergence, which is the Kullback-Leibler divergence a certain tree and the full model. In our case we can't handle the full model, we we use the $I$-divergence instead---the KL-divergence between our graph and the model where all variables are independent.

\section{information}
Self-information:
$$ I(X) = - \log(P(X)) $$
Entropy:
$$ H(p) = \E[I(p)] $$
Cross-entropy:
$$ H(p,q) = \E_p[-\log q] = H(p) + D_\mathrm{KL}(p, q). $$
\section{graph weights}
Our models use the mutual information:
$$ I(X,Y) = D_\mathrm{KM}(P(X,Y), P(X)P(Y)). $$

chi sq test on tree vs indendent graph.

\section{plotting}
Once we have sorted our edges with regard to their weight, we can plot the
graph weight vs number of parameters

\section{specific weights}
To find the weight between two models we can use a LR-test for independence.
Note we assume that two groups have the same variance, when comparing normal
and discreet variables.

\section{Test strength of graph}
Even for independent models, our method would find a forest. To see how much information our model has, we can use a $\chi ^2$-test between our model, and
the independent model.

\section{Adding Edges}
If a tree model holds, the mutual information of two variables should be the 
sum of edges along the path between them. By comparing the path with the empirical information, we can see which edges we should should add to our graph.


\end{document}
    