\documentclass[12pt]{article}
\usepackage{amsmath}

\newcommand{\E}{\textbf{E}}
\newcommand{\N}{\textbf{Norm}}

\setlength\parskip{1em}
\setlength\parindent{0em}

The G-test useful since $G = 2N * I(X,Y)$.

\begin{document}
{\Large Mar 2017}
\section{Kullback-Leibler}
So the Kullback-Leibler divergence is the expected log-likelihood ratio.
This relates to the likelihood ratio test, which uses the test-statistic
$$ D = - 2 \log \frac{L_0}{L_a},$$
which is $\chi^2$-distributed with $df_0 - df_a$ degrees of freedom.

\section{Discrete}
To test independence of two multinominal variables we use Pearson's $\chi^2$-test. 

https://en.wikipedia.org/wiki/Pearson\%27s\_chi-squared\_test

\section{Mixed}
To calculate the mutual information between continuous and discrete variables, we use a one-way analysis of variance. This mean calculating a F-test. 

The F-test give a test value, for the independence of the two variables. We still, however, need to find the mutual information.

See \emph{E Hansen 2012, Chapter 12, Introduktion til Matematisk Statistik}.

\section{Skype meeting with Steffen}
One problem I have, is how one earth do I compare continuous-continuous
mutual information, with continuous-discrete, or discrete-discrete.

Steffen talked about using the p-values for testing independence, these should be uniformly distributed under the null-hypothesis. So we can make something similar to a QQ-plot for our edges.

It's important to keep in mind how many parameters---or degrees-of-freedom--- each mutual information has, so we can penalize correctly. 
\begin{itemize}
\item continuous-continous: 1, corresponding to $\rho$ is the covariance matrix.
\item discrete-discrete: $(|F_x|-1)(|F_y|-1)$
\item continuous-discrete: (|F_x|-1).

So this allows us to compare weights that are the same type---fx. continuous-discrete. You can compare across types by using the p-value, but I'm not quite sure how to penalize and compare when choosing edges.
\end{itemize}

\end{document}

