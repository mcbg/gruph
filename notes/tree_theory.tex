\documentclass[12pt]{article}
\usepackage{amsmath}

\newcommand{\E}{\textbf{E}}
\newcommand{\N}{\textbf{Norm}}

\setlength\parskip{1em}
\setlength\parindent{0em}

\begin{document}
{\Large Mar 2017}
\section{Mutual information}
The mutual information
$$ I(x,y) = \E \left(\log \frac{ f(x,y) }{ f(x) f(y) }\right), $$
for a multivariate normal distribution is 
$$ I_\N(x,y) = - \log( 1 - \rho_{xy}^2),$$
where $\rho$ is the empirical correlation. We make a tree with the mutual information as weights, so that 
$$ W(\hat T) = \sum_{ab\in T.E} \left\{ - \log(1 - \hat\rho_{ab}^2 \right)\},$$
which is equal to the likelihood function, $\hat L(T)$. 

To fit a forest instead, we could penalize the the likelihood function with $\lambda$. Resulting in 
\begin{align}
W(\hat F) &= \sum_{ab\in F.E} \left\{ - \log(1 - \hat\rho_{ab}^2) \right\} - \mathrm{\# param}\cdot \lambda \\
&= \sum_{ab\in F.E} \left\{ - \log(1 - \hat\rho_{ab}^2) - \lambda \right\} 
\end{align}
This corresponds to putting a threshold on the mutual information of our edges.

\section{Our graphs}
In general, we will start out with a preliminary graph, $G_0$, which has the 
edges we want to include in our analysis. This could be 

\begin{itemize}
\item The full graph---containing all possible edges.
\item A graph containing all edges above a certain threshold.
\item A graph containing the $N$ largest edges.
\end{itemize}

We can impose a structure on this preliminary graph. So we get a new graph, $G_s$, which is for example

\begin{itemize}
\item a tree
\item a forest
\item a chordal graph
\end{itemize}
\end{document}
